\begin{table}

\caption{\label{tab:mix-optimality-table}Power and Type I Error Rate with Mixed Optimality}
\centering
\fontsize{9}{11}\selectfont
\begin{threeparttable}
\begin{tabular}[t]{lrrrrr}
\toprule
\multicolumn{1}{c}{ } & \multicolumn{5}{c}{Proportion High Optimality} \\
\cmidrule(l{3pt}r{3pt}){2-6}
specification & 0.1 & 0.3 & 0.5 & 0.7 & 0.9\\
\midrule
\addlinespace[0.3em]
\multicolumn{6}{c}{\textbf{Power}}\\
\hspace{1em}demand & 0.88 & 0.92 & 0.96 & 0.99 & 1.00\\
\hspace{1em}performance 1 & 1.00 & 1.00 & 1.00 & 0.99 & 0.91\\
\hspace{1em}performance 2 & 0.33 & 0.30 & 0.30 & 0.36 & 0.48\\
\hspace{1em}performance 3 & 0.35 & 0.33 & 0.34 & 0.41 & 0.56\\
\addlinespace[0.3em]
\multicolumn{6}{c}{\textbf{Type I}}\\
\hspace{1em}demand & 0.04 & 0.07 & 0.07 & 0.08 & 0.07\\
\hspace{1em}performance 1 & 0.15 & 0.16 & 0.20 & 0.25 & 0.39\\
\hspace{1em}performance 2 & 0.19 & 0.24 & 0.33 & 0.39 & 0.50\\
\hspace{1em}performance 3 & 0.20 & 0.26 & 0.34 & 0.41 & 0.50\\
\bottomrule
\end{tabular}
\begin{tablenotes}
\item \textit{Note: } 
\item The samples are constructed as a mix of a high optimality sample ($O=32$) and a low optimality sample ($O=2$). The proportion of the high optimality sample is given in the table. The remaining parameters are the same as in the baseline simulation in Figure \ref{main}
\end{tablenotes}
\end{threeparttable}
\end{table}